\chapter{Fazit}

Bei diesem Versuch durfte einiges über das Rasterkraftmikroskop gelernt werden.
Mit teils bekannter und teils unbekannter Theorie, konnte nun das Prinzip einer Bildlichen Darstellung in fast Atomarer Auflösung verstanden werden.
Es wurden verschiedene Topographien von einer Kalibrierungsprobe aufgenommen, um damit zu testen, ob das Mikroskop richtig kalibriert war.
Schön zu sehen war, wie sich die Abbildung durch die Forwärts, bzw. Rückwertsbewegung ändert.
Dann wurde eine Resonanzkurve des Cantilevers aufgenommen, um die resonanzfrequenz zu erhalten.
Unsere Messwerte stimmten sehr exakt mit denen des Herstellers überein, minimale Fehler sind mit diesem Aufbau, wegen Luftverwirbelungen und Erschütterungen, unvermeidbar.
Danach wurde die Topographie einer CD ausgelesen, um deren Kapazität zu stoßen.
Auch hier war unser Ergebnis ohne Einwand zu aktzeptieren.
Es deutet nichts darauf hin, dass unser Versuch Probleme gehabt hätte.
