\chapter{Vorbereitung}

\section{Theoretische Grundlagen zum AFM}
       
       
 \subsection{Kräfte zwischen Atomen}
        \subsection{Van-der-Waals Kräfte}

Die Ladungsverteilung in Atomen ist nicht konstant, sondern unterliegt ständiger 
Fluktuation. Der Schwerpunkt der negativen Ladungen kann hierbei vom dem der
positiven Ladungen abweichen. Ist dies der Fall, so entsteht ein Dipol. 
Befindet sich nun ein zweites Teilchen in der Nähe dieses Atoms, so wird auch
in diesem ein Dipol induziert. Zeigt die positive Seite des ersten Atoms zu Atom 2,
so werden die Elektronen des zweiten Atoms angezogen. Ist es die negative Seite, so
werden die Elektronen abgestoßen. 
\vspace{3pt}\\
Als Folge dessen synchronisieren sich die Ladungsänderungen der beiden Atome. Eine
schwache positive Anziehung ist die Folge. Diese ist proportional zu $\displaystyle
\frac{-1}{r^6}$.

        \subsection{Pauli-Abstoßung}

Nähern sich die Atome weiter an, so kommt es zu einem Überlappen der 
Elektronenorbitale. Das Pauli-Verbot verhindert hierbei, dass zwei Elektronen den
gleichen Zustand besetzen. Einige Elektronen werden folglich in einen energetisch
höheren Zustand gezwungen. \\
So führt eine Orbitalüberlagerung zu einer repulsiven Wechselwirkung. Die Kraft
ist proportional zu $\displaystyle \frac{1}{r^{12}}$.

        \subsection{Lennard-Jones Potential}

Bei sehr kleinen Abständen dominiert die Pauli-Abstoßung, bei größeren die 
van-der-Waals Kräfte. Die Summe aus beiden Potentialen wird Lennard-Jones 
Potential gennant. 
\[
   \phi (r) \propto \frac{A}{r^6} - \frac{B}{r^{12}}    
\]

Dabei bezeichnet $\phi$ das Potential und somit die Bindungsenergie, r den Abstand.
A, B sind Konstanten die stoffspezifisch sind.

\begin{figure}[h!]
    \centering
    \includegraphics[width=0.6\textwidth]{Abb/ljp.jpg}
    \caption{Das Lennard-Jones Potential als Summe der vdW-Wechselwirkung und
             der Pauli-Abstoßung}
    \label{ljp}
\end{figure}

Neben diesen Kräften können im Allgemeinen auch noch chemische Bindungskräfte, Kontaktkräfte, magnetische und elektrische Wechselwirkungen eine Rolle besitzen.
Bei unserem Aufbau haben sie jedoch nur geringe Bedeutung.



\subsection{Der Cantilever}
\label{herleitung}

 
Der Cantilever ist ein schwingungsfähiger Balken, der eine pyramidale Spitze mit einer Dicke von nur wenigen Nanometern besitzt.
Er wird meist aus $Si_3N_4$ hergestellt und an dessen Ende wird, durch Ätzung, eine abstehende sehr sehr dünne Spitze geformt.
Seine Resonanzfrequenz befindet sich etwa im kHz bis MHz Bereich.
Ihm wird die größte Bedeutung in diesem Versuch zugesprochen, denn mit ihm lassen sich nun die bereits besprochenen Abstoßungs- bzw. Anziehungskräfte messen.
Der Cantilever, verhält sich durch seine periodische Bewegung in guter Näherung wie ein getriebener, gedämpfter harmonischer Oszillator. 
Die Formel dazu sieht folgendermaßen aus:
\[
    m \ddot{x} + \frac{m \omega_0}{Q} \dot{x} + kx = F_0 \cos(\omega t)
\]
Wir verwenden hier m als die punktförmig genäherte Masse des Cantilevers, $\omega_0$ ist dessen Eigenfrequenz mit seiner Güte $Q$ und $k$ beschreibt eine Federkonstante die für die rücktreibende Kraft, also die Oszillation, verantwortlich ist.
Auf der rechten Seite der Gleichung beschreibt $F_0$ die treibende Kraft, die die Probe und die Vorrichtung auf den Cantilever wirken.
Wir benutzen den Ansatz:
\[
    x(t)=A \cdot e^{iwt}
\]
und bekommen schließlich durch Einsetzen folgende Gleichung.
\[
   \left(-\omega^2+\omega_0^2-i\frac{\omega_0}{Q} \cdot \omega \right) \cdot A 
   = \frac{F_0}{m}
\]
Löst man diese Gleichung anschließend nach der Amplitude A auf, erhält man 
folgendes Ergebnis:
\[
    A = \frac{F_0}{m \sqrt{ ( \omega_0^2 - \omega^2 )^2 + \left( \frac{\omega 
        \omega_0}{Q} \right)^2}}
\]

Die Auswirkungen auf seine Resonanzkurve bei attraktiver und repulsiver Wechselwirkung lassen sich der Abbildung ... entnehmen.





       
 \section{Aufbau des Rasterkraftmikroskops}
\begin{figure}[h!]
    \centering
    \includegraphics[width=0.6\textwidth]{Abb/afm.jpg}
    \caption{Aufbau eines AFM}
    \label{afm}
\end{figure}

Das Mikroskop besteht aus einem Cantilever mit Messspitze, dieser wurde bereits erläutert, einem Positioniersystem für die z-Richtung, einem Positioniersystem für x- bzw. y-Richtung und einer Detektionseinheit, welche die Amplitudenänderung des Cantilevers misst. 
Diese Bauteile sollen in den folgenden Kapiteln noch genauer beschrieben werden.  
 Man fährt den Cantilever mithilfe der Rastereinheit in nächster Nähe über die Probe.
Die Spitze wird mittels Schrittmotoren auf einen ungefähren Abstand von einem Mikrometer angenähert und anschließend über piezoelektrische Bauelemente weiter ausgerichtet.
Diese Vorrichtung führt letztendlich die Spitze über die Probe, in Entfernungen von 10 - 100 Mikrometern in x- und y-Richtung und einer Höhe von unter 10 Mikrometern.
Gleichzeitig versetzt man den Cantilever nahe seiner Resonanzfrequenz und beobachtet über die Detektionseinheit, wie stark diese Schwingung durch den Abstand zur Probe eingeschränkt wird.

 \subsection{Detektionseinheit}
 
 Das in diesem Versuch verwendete EasyScan DFM Rasterkraftmiskroskop nutzt zur Auslesung der Amplitudenänderung ein optisches Verfahren. 
 Man verwendet einen Laser, der auf die Rückseite des Cantilevers ausgerichtet ist, wo sich eine polierte Stelle befindet, welche das Licht des Lasers spiegelt.
 Bei der Deformation des Cantilevers ändert sich somit die Position des gespiegelten Lichtbündels.
 Mithilfe einer Photodiode lassen sich diese minimalen Änderungen sehr gut messen und in Größenordnungen von einzelnen Angstrøm umrechnen.
 Diese Methode eignet sich hervorragend, denn das Laserlicht wird lediglich durch Erschütterungen gestört.
 Deshalb baut man diesen Versuch auf einem massiven Steintisch auf und versucht Erschütterungen bei der Messung zu vermeiden.

\subsection{Rastermechanismus}

Nun muss noch beschrieben werden, wie es sich mit der Rastereinheit verhält.
Am Anfang fährt man die Probe mittels eines Schrittmotors mechanisch bis auf wenige Mikrometer auf die Probe in z-Richtung heran.
Anschließend nutzt man den piezoelektrischen Effekt um eine feinere Ansteuerungen zu ermöglichen.
Zumeist verwendet man piezoelektrische Röhrchen aus Blei-Zirkonat-Titanat, denn dieses Material kann sich mithilfe einer angelegten Spannung stark dehnen und zusammenziehen. 
Um den piezoelektrischen Effekt und sein Inverses korrekt verstehen zu können, muss man auf atomare Ebene dieser Moleküle nach den Ladungsverschieben schauen.



edit soon ------------------------------------- (piezo)



Typische Rasterbereiche sind $10-\SI{100}{\mu m}$ in x- und y-Richtung und $2-\SI{5}{\mu m}$ in z-Richtung.
Das hier verwendete Mikroskop basiert auf diesem elektromechanischem Prinzip, also der Deformierung der piezoelektrischen Materialien zur feineren Bewegung des Cantilevers.
Alternativ lässt sich auch die Probe bewegen.


\section{Betriebsmodi}


Es gibt zwei Methoden um die Kräfte der Probe auf den Cantilever wirken zu lassen.
Es gibt den statischen Modus und den dynamischen Modus. 
Beim statischen Modus wird der Cantilever nicht in Schwingung versetzt und in Ruhe gehalten.
Der dynamische Modus arbeitet hingegen mit einem schwingenden Cantilever.
Der statische Modus ist leichter umzusetzen und war desshalb zu Beginn der Rasterkraftmikroskopie die vorherrschende Technik, heutzutage verwendet man eher die dynamischen Modi, welche keinen physischen Kontakt mit der Probe haben.


\subsection{Statischer Modus:}

Bei dem statischen Betriebsmodus bringt man die Messspitze des Cantilevers in direkten Kontakt mir der Probe.
Dadurch verbiegt sich dieser entsprechend attraktiv und repulsiv, was man wie schon erläutert, in eine Messreihe umwandeln kann.
Der Rastermechnismus erhält diese Information ebenso, sodass er so nachgeregelt werden kann, dass er die Probe eben genau berührt.
Man kann allerdings auch die Höhe des Cantilevers über der Probe konstant halten und die Wechselwirkung mit der Probe aufzeichnen.
Der statische Modus hat jedoch Nachteile.
Der Cantilever und die Probe werden dabei verändert, oder sogar zerstört.
Auch das Nachregeln der Höhe, kann bei großen Bergen und Tälern die Spitze direkt in die Probe "rammen".
Lässt man den Cantilever auf konstanter Höhe, können die Wechselwirkungen zur Probe zu gering sein, um sie messen zu können.

Weil sich bei diesem Modus die Bauteile sehr stark abnutzen entwickelte man eine Alternative, den dynamischen Modus.

\subsection{Dynamischer Modus:}

       \paragraph{Amplitudenmoduliertes Rasterkraftmikroskop}
 
 
       \paragraph{Frequenzmoduliertes Rasterkraftmikroskop}
Dazu versetzt man den Cantilever, nach dem Prinzip der harmonischen Schwingung, in die Nähe seiner Eigenresonanz.
Rege man den Cantilever bei seiner Eigenfrequenz an, würde sowohl attraktive als auch repulsive Kraftwechselwirkung zu einer Absenkug von Amplitude oder Frequenz führen. Dies wird in der Graphik ... veranschaulicht.
Um einfache Ergebnisse erzielen zu können, kann man einerseits den Cantilever ständig nahe seiner Resonanzfrequenz versetzen und die Änderung der Schwingungsamplitude messen, oder eine Phasenverschiebung der Anregungsfrequenz mit der Antwort des Cantilevers abgleichen.
Mit diesem eindeutigen Ergebnis lässt sich anschließend die Topographie der Probe rekonstruieren.
\vspace{6pt}\\
In unserem Versuch wird die Methode der Amplitudenänderung verwendet. 
Die anziehenden und abstoßenden Kräfte deformieren den Cantilever so stark, dass der Weg des Lasers im Angström Bereich umgelenkt wird. 
Eine Amplitudenänderungen von 70\%, zur Resonanzamplitude, ist maximal. 


Edit über AM und FM   ---------------------------------------------

 

